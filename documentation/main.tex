\documentclass{documentation}

\showhelp  % comenta o borra para eliminar ayudas

\selectlanguage{spanish}

%TITLE
\title{wearControl}
\author{Josue Gutierrez Duran}
\docdate{2017}{Diciembre}

\begin{document}

\maketitle
\tableofcontents

\newpage

\section{DESCRIPCIÓN GENERAL}


Los smartwatch son dispositivos electrónicos inteligentes que están claramente en auge. Gracias a ellos es posible extender funciones de los dispositivos móviles u ordenadores. Dado que en la asignatura de referencia \textit{Multimedia} se han tratado temas referentes a Audio y Vídeo. Se propone realizar una aplicación wear que sea capaz de controlar los controles multimedia de un ordenador. Esta aplicación sera independiente de un dispositivo móvil, no siendo necesario este para funcionar. Por lo que sera necesario que el smartwatch tenga conexión Wi-Fi. 
\newline
\par
\noindent
Para la realización de este proyecto se desarrollaran dos aplicaciones distintas, una primera aplicación que se ejecutara en el ordenador que se desea controlar, y una segunda aplicación ejecutada en el smartwatch que actuara como mando a distancia.

\section{REQUISITOS DE LA APLICACIÓN}

Para este trabajo, se deben de cumplir algunos de los requisitos propuestos en la documentación de la asignatura y adjuntados en el \textbf{Cuadro 1}. Por tanto, se proponen los requisitos de aplicación registrados en el \textbf{Cuadro 2}.

\begin{table}[hp]
  \centering
  \caption{Requisitos Generales del Trabajo}
  \label{tab:tec-especifica}

  \zebrarows{1}
  \begin{tabular}{p{0.6\textwidth}}
    \hline
    Comunicación \\
    Descarga, actualización \\
    Visualización y/o reproducción \\
    Interacción con usuario y/o con otros sistemas Hw/Sw \\
    Compresión \\
    Procesamiento de señales \\
    \hline
  \end{tabular}
\end{table}


\begin{table}[hp]
  \centering
  \caption{Requisitos de la Aplicación wearControl}
  \label{tab:competencias}

  \zebrarows{1}
  \begin{tabular}{p{1\textwidth}}
    \hline
    La aplicación wear debe de ser independiente de cualquier otro dispositivo\\
    La conexión entre PC y Android Wear debe de ser a traves de una red Wi-Fi\\
    Deben de poderse conectar mas de un sistema en una misma red local\\
    Debe de ser posible controlar desde el smartwatch los controles multimedia, tales como Play, Stop, Back y Next.\\
    Debe de ser posible controlar los controles de Volumen; Up, Down y Mute. \\
    Debe de ser posible abrir aplicaciones multimedia en el PC, Spotify, VLC y iTunes \\
    \hline
  \end{tabular}
\end{table}

\newpage
\section{DECISIONES Y JUSTIFICACIÓN DE DISEÑO E IMPLEMENTACIÓN}

- JUSTIFICACION -


\section{CONCLUSIONES}

- CONCLUSIONES -

\section{BIBLIOGRAFÍA}

Como Bibliografía utilizada en este trabajo, tenemos diferentes paginas web y documentación referente al diseño de aplicaciones Android Wear, a continuación se nombran todas las referencias utilizadas.

\begin{itemize}
\item Android Web Developers - \url{https://developer.android.com}
\item Smartwatch Tutorial - Android Wear for Beginners - \url{http://www.smartwatch.me/t/tutorial-how-to-develop-android-wear-apps-for-beginners-part-1-setup/684}
\end{itemize}


\section{MANUAL DE USUARIO}

- MANUAL DE USUARIO -


\end{document}


% Local Variables:
% coding: utf-8
% mode: flyspell
% ispell-local-dictionary: "castellano8"
% mode: latex
% End:
